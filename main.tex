\documentclass{article}
\usepackage[utf8]{inputenc}
\usepackage[margin=1in, top=0.8cm]{geometry}
\usepackage{amsmath}
\usepackage{amssymb}
\usepackage{comment}

\usepackage{url}
\usepackage{pdfpages}

\usepackage{eurosym}
% Let's us use € to get the \EUR command from the eurosym package.
\DeclareUnicodeCharacter{20AC}{\EUR}

\usepackage{listings}% http://ctan.org/pkg/listings
\lstset{
  basicstyle=\ttfamily,
  mathescape
}
\lstMakeShortInline[columns=fixed]|

\title{Meet your group \\ \large MFV240V Cyber-Physical Systems Safety and Security \\ Group B}

\author{
  \and
  Muhammad Rusyadi Ramli\\
  \texttt{ramli2@kth.se}
  \and
  Shamoona Imtiaz\\
  \texttt{shamoona.imtiaz@mdh.se}
  \and
  Konstantinos Kalogiannis\\
  \texttt{konkal@kth.se}
  \and
  Javier Cabrera Arteaga\\
  \texttt{javierca@kth.se}
  \and
  Marcus Schmidt Birgersson\\
  \texttt{marbir@kth.se}
}

\date{\today}

\usepackage{natbib}
\usepackage{graphicx}



\begin{document}

\maketitle


\section{What are your skills and strengths?}

\textit{Konstantinos Kalogiannis:} Currently, my research concerns the security of Intelligent Transportation Systems. My skills are focused in system development and network security.

\textit {Shamoona Imtiaz:} My skills are mostly in area of programming and information security. Currently I am working on profiling of low-level resources and fingerprinting of applications for detection purposes.

\textit{Javier Cabrera Arteaga (Phd Student at SCS department, KTH EECS school):} My main research topic is on Automatic Software Diversification, for reliability and security. Software Diversification can be used for testing and therefore to harden systems.

\textit {Marcus Birgersson:} My research is regarding cyber security with focus on Internet of Things. My skills and interests are foremost towards software development.

\textit {Muhammad Rusyadi Ramli:} My research topic is about reference architecture for trustworthy (safe and secure) collaborative Cyber Physical Systems. My skills are mostly in area of systems engineering and systems architecting.

\section{General work guidelines}

The team will maintain a GitHub repository (\url{https://github.com/Jacarte/MF240VHT211}) where all the progress and work will be saved. The main goal of the repository is to facilitate the progress on the course tasks. It will also help us maintain a tight communication and to easily share our results. 

\subsection{How do you want to communicate?}

The team will maintain mostly an asynchronous communication; the GitHub Issue Tracker will be used as the main channel. Besides, email communication will be used as a secondary channel. Finally, for the live meetings we opted to utilize zoom (\url{https://kth-se.zoom.us/j/63556585720}).

\subsection{How will you keep track of open tasks?}

In general, the tasks will be managed using the GitHub Issue Tracker, e.g. each new task has a new issue. The tasks should be labelled with semantic short names, for example, for open tasks, the label "open" should be used. Notice that a task could have more than one label and that not all open issues are for tasks (for example code bugs). 

Every issue should have an assignee, which in this case will be the leader of the task. The tasks/issues should be closed after they are finished. Each task should have a general description. Discussion on the task should be reflected as comments inside the issue. The last comment of the issue should be a link to the report or repository folder containing the deliverable.


\subsection{How do you plan the work?}

Every task should follow the next steps from its creation to its successful assessment:

\begin{enumerate}
    \item Create a new issue in the repository Issue Tracker. The title of the task should be the same name as the Canvas assignment.
    \item The description of the tasks should start with the link to the assignment in Canvas. Next, the full description of the task should be added (if it is too large, a summary should be added).
    \item An assignee is added to the issue. The assignee will be the leader of the task.
    \item The issue is labeled as "open".
    \item Every new important change in the task, such as new added assets, should be reflected in a comment inside the issue. 
    \item The issue is completely open for team discussion.
    \item When the task is finished, the last comment should be a link to the deliverable.
    \item The issue is closed.
\end{enumerate}

% 0 - Register the assignment goals in the tracking system
% 1- Write the assignment goals as the document/repo_README outline
    % 1.2- Update this created documentation every time a tracked issue is assesed
% 2 - The final report should be based on the updated documentation.
    

\subsection{How will you create the presentations/reports?}

There is total freedom to use whatever technology is available to each member to prepare the presentations and the reports. However, every created asset should be saved as a file in the repository in order to track the progress and have everything gathered in one location.


\subsection{Are you going to assign a leader for each assignment, who will coordinate and manage the tasks?}

The assignment of the leader (issue assignee) will be managed based on the most needed skills for the task.

\section{Code of conduct}

\begin{itemize}
    \item Each member of the team should use its skills and contribute equally to the completion of the tasks required.
    \item This document is a general guideline for the team working. It will be updated regularly depending on the needs and the team discussions.
\end{itemize}




\pagenumbering{gobble}

\bibliographystyle{plain}
%\bibliography{references}
\end{document}
